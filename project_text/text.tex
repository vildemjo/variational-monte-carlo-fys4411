\section{The derivatives and the local energy}

In order to find the drift force and the local energy analytically we need to calcualte both the derivative and the double derivative of the trial wavefunction.

\subsection{The derivative of the trial wave function}

We separete the total trial wave function (Eq. \ref{eq:trialwf}) into the onebody part and the interaction part,
\begin{equation}
\Psi_T = \Psi_{ob}\Psi_{in}.
\end{equation}

Using the product rule, the derivative with regards to the particle $k$ is
$$ \nabla_k \Psi_T =  \Psi_{ob}\nabla_k\Psi_{in} + \nabla_k\Psi_{ob}\Psi_{in}.$$
(Here the operator $\nabla_k$ only works on the first function after it.)

So we have to calculate $\nabla_k\Psi_{ob}$ and $\nabla_k\Psi_{in}$ and insert the expressions into the equation above.

We have (if $g(\mathbf{r}_k,\alpha) = \phi(\mathbf{r}_k)$)
\begin{equation}\label{eq:psi_ob_derivative}
\nabla_k\Psi_{ob} = \nabla_k \phi(\mathbf{r}_k)\prod_{k\neq i}^N \phi(\mathbf{r}_i)= \frac{\nabla_k \phi(\mathbf{r}_k)}{\phi(\mathbf{r}_k)} \Psi_{ob}
\end{equation}
using the chain rule.

The interaction part is a little more complicated. We start with
$$ \nabla_k\Psi_{in} = \nabla_k  \exp{\left(\sum_{j<i}u(r_{ji})\right)} = \exp{\left(\sum_{j<i}u(r_{ji})\right)} \sum^N_{l \neq k}  u\left(r_{kl}\right) \nabla_k u (r_{kl}) $$. Because it is an exponential function we have to multiply the original function with the sum over all the terms in the exponent that are dependant on the particle $k$. Because $u(r_{kj}) = u(r_{jk})$ this sum is $\sum^N_{l \neq k}  u\left(r_{kl}\right)$. In addition we have to multiply with the derivative of $u(r_{kl})$ because of the chain rule. Since we have a sort of simplified way of showing the derivative (using the operator $\nabla_k$), the expression $u(r_{kj})\nabla_k u(r_{kj}) = \nabla_k u(r_{kj})$ by using the chain rule the opposite way. We then have the expression

\begin{equation}\label{eq:psi_in_derivative}
\nabla_k\Psi_{in} = \exp{\left(\sum_{j<i}u(r_{ji})\right)} \sum^N_{l \neq k} \nabla_k u (r_{kl}) = \sum^N_{l \neq k} \nabla_k u (r_{kl}) \Psi_{in}
\end{equation}
for the derivative of the interaction part of the wave function.

The total expression of the derivative of the trial wave function is hence 
\begin{align}
\nabla_k \Psi_T &= \Psi_{ob}\nabla_k\Psi_{in} + \nabla_k\Psi_{ob}\Psi_{in}\\
 &= \prod_{k\neq i}^N \phi(\mathbf{r}_i)\exp{\left(\sum_{j<i}u(r_{ji})\right)} \sum^N_{l \neq k} \nabla_k u (r_{kl}) + \nabla_k \phi(\mathbf{r}_k)\prod_{k\neq i}^N \phi(\mathbf{r}_i) \exp{\left(\sum_{j<i}u(r_{ji})\right)}\\
 &= \Psi_{ob}\sum^N_{l \neq k} \nabla_k u (r_{kl}) \Psi_{in}+ \frac{\nabla_k \phi(\mathbf{r}_k)}{\phi(\mathbf{r}_k)} \Psi_{ob}\Psi_{in}\\
 &= \left(\sum^N_{l \neq k} \nabla_k u (r_{kl})+ \frac{\nabla_k \phi(\mathbf{r}_k)}{\phi(\mathbf{r}_k)}\right) \Psi_T 
\end{align}

\subsection{The double derivative of the trial wave function}

The double derivative with regards to particle $k$ is

\begin{equation}\label{eq:total_double_start}
nabla^2_k \Psi_T =  \Psi_{ob}\nabla^2_k\Psi_{in} + 2\nabla_k\Psi_{ob}\nabla_k\Psi_{in} + \Psi_{ob}\nabla^2_k\Psi_{in}.
\end{equation}
So we have to calculate $\nabla^2_k\Psi_{ob}$ and $\nabla^2_k\Psi_{in}$, in addition to the derivatives form the prevoius section, and insert the expressions into the equation above.

From Eq. \ref{eq:psi_ob_derivative} we find
$$ \nabla^2_k\Psi_{ob} = \nabla^2_k \phi(\mathbf{r}_k)\prod_{k\neq i}^N \phi(\mathbf{r}_i) = \frac{\nabla^2_k \phi(\mathbf{r}_k)}{\phi(\mathbf{r}_k)} \Psi_{ob},$$
since $\prod_{k\neq i}^N \phi(\mathbf{r}_i)$ is indepedant of the particle $k$.

For the double derivative of $\Psi_{in}$ we use the product rule again
\begin{equation}\label{eq:double_derivative_start}
\nabla^2_k\Psi_{in} = \exp{\left(\sum_{j<i}u(r_{ji})\right)} \nabla_k\left[ \sum^N_{l \neq k} \nabla_k u (r_{kl})\right] +\nabla_k\left[ \exp{\left(\sum_{j<i}u(r_{ji})\right)}\right] \sum^N_{l \neq k} \nabla_k u (r_{kl}).
\end{equation}
Here the square brakets are used to show what $\nabla_k$ applies to. So,

$$\nabla_k\left[ \exp{\left(\sum_{j<i}u(r_{ji})\right)}\right]=  \exp{\left(\sum_{j<i}u(r_{ji})\right)} \sum^N_{l' \neq k} \nabla_k u (r_{kl'})$$ as in the previous section. We calculate $ \nabla_k u(r_{kl'})$ using the chain rule and get
$$ \nabla_k u(r_{kl'})  = \frac{\mathbf{r}_k - \mathbf{r}_{l'}}{r_{kl'}} u'(r_{kl'}), $$ where $u'(r_{kl'}) = \frac{d}{dr_{kl'}}u(r_{kl'})$, because $$\frac{d}{d\mathbf{r}_{k}}r_{kl'} =\frac{d}{d\mathbf{r}_k}\sqrt{(x_k-x_{l'})^2 + (y_k-y_{l'})^2 + (z_k-z_{l'})^2} = \frac{\mathbf{r}_k - \mathbf{r}_{l'}}{r_{kl'}} $$ and then
$$\nabla_k\left[ \exp{\left(\sum_{j<i}u(r_{ji})\right)}\right]=  \exp{\left(\sum_{j<i}u(r_{ji})\right)} \sum^N_{l' \neq k} \frac{\mathbf{r}_k - \mathbf{r}_{l'}}{r_{kl'}} u'(r_{kl'}) = \Psi_{in} \sum^N_{l' \neq k} \frac{\mathbf{r}_k - \mathbf{r}_{l'}}{r_{kl'}} u'(r_{kl'})  $$

Next, we have
\begin{equation}\label{eq:double_der_part_in}
\nabla_k\left[ \sum^N_{l \neq k} \nabla_k u (r_{kl})\right] = \nabla_k \left[ \sum^N_{l \neq k}  \Psi_{in} \frac{\mathbf{r}_k - \mathbf{r}_{l}}{r_{kl}} u'(r_{kl}) \right] = \sum^N_{l \neq k} \nabla_k \left[ d \cdot \frac{a}{b} \cdot c \right]
\end{equation}
where $a = \left(\mathbf{r}_k-\mathbf{r}_{l}\right)$, $b= r_{kl}=|\mathbf{r}_k-\mathbf{r}_{l}|$ , $c = u'(r_{kl})$ and $d = \Psi_{in}= \exp\left(\sum_{i<j}^N u(r_{ij})\right)$. Furthermore, $a' = \frac{d}{d\mathbf{r}_k} \left(\mathbf{r}_k-\mathbf{r}_{l}\right) = \frac{d}{dx_k} x_k  + \frac{d}{dy_k} y_k  +\frac{d}{dz_k} z_k   = 3$ (the number of dimensions), $b' = \frac{d}{d\mathbf{r}_k} r_{kl} = \frac{\mathbf{r}_k - \mathbf{r}_{l}}{r_{kl}}$, $c' = \frac{d}{d\mathbf{r}_k} u'(r_{kl}) =  u''(r_{kl})\frac{\mathbf{r}_k - \mathbf{r}_{l}}{r_{kl}}$ and $d' = \Psi_{in} \sum^N_{l \neq k} \frac{\mathbf{r}_k - \mathbf{r}_{l}}{r_{kl}} u'(r_{kl})$. Inserting all these expression into Eq. \ref{eq:double_der_part_in} gives (skipping some of the simplifications since the explaination is in the above part)
\begin{align}
\nabla_k\left[ \sum^N_{l \neq k} \nabla_k u (r_{kl})\right] &= \sum^N_{l \neq k} \left( \left(\frac{3r_{kl}}{\left(\mathbf{r}_k-\mathbf{r}_{l}\right)^2}-\frac{1}{r_{kl}}\right)u'(r_{kl}) + u''(r_{kl})\right) \frac{\left(\mathbf{r}_k-\mathbf{r}_{l}\right)^2}{r_{kl}^2} \Psi_{in} \\ \label{eq:doble_der_u}
&= \sum^N_{l \neq k} \left( \frac{2}{r_{kl}}u'(r_{kl}) + u''(r_{kl})\right) \Psi_{in}
\end{align}
When we insert the results into Eq. \ref{eq:double_derivative_start} we get
$$\nabla^2_k\Psi_{in} = \Psi_{in}\sum^N_{l \neq k} \left( \frac{2}{r_{kl}}u'(r_{kl}) + u''(r_{kl})\right) \Psi_{in} +  \Psi_{in} \sum^N_{l' \neq k} \frac{\mathbf{r}_k - \mathbf{r}_{l'}}{r_{kl'}} u'(r_{kl'}) \Psi_{in} \sum^N_{l \neq k} \frac{\mathbf{r}_k - \mathbf{r}_{l}}{r_{kl}} u'(r_{kl})  $$

$$\nabla^2_k\Psi_{in} = \Psi^2_{in} \left[\sum^N_{l \neq k} \left( \frac{2}{r_{kl}}u'(r_{kl}) + u''(r_{kl})\right) +   \sum^N_{l' \neq k} \sum^N_{l \neq k} \frac{(\mathbf{r}_k - \mathbf{r}_{l'})(\mathbf{r}_k - \mathbf{r}_{l})}{r_{kl'}r_{kl}}  u'(r_{kl})u'(r_{kl'}) \right] $$

Incerting it all into Eq. \ref{eq:double_derivative_start} and dividing by the trial wavefunction (as we will do to find the local energy) gives
\begin{align*}
   \frac{1}{\Psi_T(\mathbf{r})}\nabla_k^2\Psi_T(\mathbf{r})
   &= \frac{\nabla_k^2\phi(\mathbf{r}_k)}{\phi(\mathbf{r}_k)}
   + 2\frac{\nabla_k\phi(\mathbf{r}_k)}{\phi(\mathbf{r}_k)}
   \left(\sum_{l\ne k}\frac{(\mathbf{r}_k-\mathbf{r}_l)}{r_{kl}}u'(r_{kl})\right)
   \\
   &\qquad
   + \sum_{l\ne k}\sum_{l' \ne k}\frac{(\mathbf{r}_k-\mathbf{r}_l)(\mathbf{r}_k-\mathbf{r}_{l'})}{r_{kl}r_{kl'}}u'(r_{kl})u'(r_{kl'})
   \\
   &\qquad
   + \sum_{l\ne k}\left( u''(r_{kl})+\frac{2}{r_{kl}}u'(r_{kl})\right).
\end{align*}

\section{The local energy and drift force as implemented in the code}\label{sec:implementation}

To calculate the kinetic energy part of the local energy we use the last expression in the previous section, summed over all particles $k$. We use $\phi(\mathbf{r}_k)$ from Eq. \ref{eq:phi} and find 

$$ \sum_k^N\frac{\nabla_k^2\phi(\mathbf{r}_k)}{\phi(\mathbf{r}_k)} = -2\alpha Nd + 4 \alpha^2 \sum_k^N \mathbf{r}_k^2 $$ where $d$ is the number of dimensions and $\mathbf{r}_k^2 = x_k^2 + y_k^2+\beta z_k^2$. This is the expression for the kinetic part of the local energy if there is no interaction. Furthermore, from Eq. \ref{eq:phi} and Eq. \ref{eq:psi_ob_derivative}

$$ \frac{\nabla_k\phi(\mathbf{r}_k)}{\phi(\mathbf{r}_k)} = -2\alpha  \mathbf{r}_k $$
 
We also have
\begin{align*}
u'(r_{kl}) &= -\frac{a}{ar_{kl}-r_{kl}^2} \text{ and }\\
u''(r_{kl}) ) &= \frac{a(a-2r_{kl})}{r_{kl}^2(a-r_{kl})^2}
\end{align*} for $r_{kl} > a$. The other case is not relevant because the local energy is never sampled if $r_{kl} < a$. Then the wave equation is zero. With this the kinetic part of the local energy can be calculated analytically with our choice of trial wavefunction.

To get the potential energy part of the local energy the sum over all particles $k$ is made with the relevant expression for the trap from Eq. \ref{eq:trap_eqn}. This is done for both the analytical and the numerical evaluation of the local energy.

The drift force, $F$, given by Eq. \ref{eq:drift_force} and by extracting the relevant expressions from the equations above we get
$$ F(\mathbf{r}_k) = 2\left(\frac{\nabla_k\phi(\mathbf{r}_k)}{\phi(\mathbf{r}_k)}
   +\sum_{l\ne k}\frac{(\mathbf{r}_k-\mathbf{r}_l)}{r_{kl}}u'(r_{kl})\right) = 2 \left( -2\alpha  \mathbf{r}_k  - \sum_{l\ne k}\frac{(\mathbf{r}_k-\mathbf{r}_l)}{r_{kl}}\frac{a}{ar_{kl}-r_{kl}^2}\right) $$ 
where the last term in the paranthesis is removed when there is no interaction.


\subsection{Brute force sampling calculations of the energy with various $\alpha$ and number of particles}\label{app:alpha_lists_brute_force}

\begin{table}[H]\caption{50 particles}\label{tab:brute_force_N_50}
\begin{tabular}{lllll}
$\alpha$: & $\left< E_L \right>$:& $E_{exact}$ & $\sigma_B$ & $\sigma$\\ \hline
0.35 & 79.49535 & 79.82143 & 0.23481 & 2.97403\\
0.40 & 77.08187 & 76.87500 & 0.12685 & 1.94938\\
0.45 & 75.34621 & 75.41667 & 0.05503 & 0.88031\\
0.50 & 75.00000 & 75.00000 &                &                \\ 
0.55 & 75.20971 & 75.34091 & 0.05544 & 0.86965\\
0.60 & 76.10958 & 76.25000 & 0.09233 & 1.57130\\
0.65 & 77.71489 & 77.59615 & 0.13609 & 2.33112\\
\end{tabular}
\end{table} 

\begin{table}[H]\caption{100 particles}\label{tab:brute_force_N_100}
\begin{tabular}{lllll}
$\alpha$: & $\left< E_L \right>$:& $E_{exact}$ & $\sigma_B$ & $\sigma$\\ \hline
0.35 & 160.41867 & 159.64286 & 0.42227 & 4.47971\\
0.40 & 153.99383 & 153.75000 & 0.25800 & 2.78924\\
0.45 & 150.84608 & 150.83333 & 0.08956 & 1.23005\\
0.50 & 150.00000 & 150.00000 &                 &                \\ 
0.55 & 150.67186 & 150.68182 & 0.11373 & 1.26926\\
0.60 & 152.53009 & 152.50000 & 0.18938 & 2.15082\\
0.65 & 155.05236 & 155.19231 & 0.24124 & 2.99347\\
\end{tabular}
\end{table} 

\begin{table}[H]\caption{500 particles}\label{tab:brute_force_N_500}
\begin{tabular}{lllll}
$\alpha$: & $\left< E_L \right>$:& $E_{exact}$ & $\sigma_B$ & $\sigma$\\ \hline
0.35 & 790.86683 & 798.21429 & 0.42590 & 4.30332\\
0.40 & 762.88550 & 768.75000 & 0.40947 & 3.79362\\
0.45 & 751.76787 & 754.16667 & 0.14928 & 1.47760\\
0.50 & 750.00000 & 750.00000 &                 &                \\ 
0.55 & 755.78734 & 753.40909 & 0.13751 & 1.38974\\
0.60 & 766.83482 & 762.50000 & 0.23418 & 2.49695\\
0.65 & 782.69730 & 775.96154 & 0.42778 & 4.21930\\
\end{tabular}
\end{table} 

\begin{table}[H]\caption{The calculated energies, $\left<E_L\right>$, for one particle in three dimensions compared with the exact energy, $E_{ex}$. Both energies are of units $\hbar\omega_{oh}$. These calculations were performed with $2^{24}$ number of MC cycles. The normal standard deviation $\sigma$, and the variance from the blocking resampling method, $\sigma_B$ are also included. }\label{tab:brute_force_N_1_MC_boost}
\center
\begin{tabular}{cccccc}
$\alpha$ & $\left< E_L \right>$ & $E_{ex}$ & |$\left< E_L \right>-E_{ex}$|  & $\sigma_B$ & $\sigma$\\ \hline
0.35 & 1.59627 & 1.59643 & 0.00016 & 0.00552 & 0.44298\\
0.40 & 1.53763 & 1.53750 & 0.00013 & 0.00333 & 0.27608\\
0.45 & 1.50793 & 1.50833 & 0.00040 & 0.00138 & 0.12657\\
0.50 & 1.50000 & 1.50000 &                &                &                 \\
0.55 & 1.50516 & 1.50682 & 0.00165 & 0.00119 & 0.11845\\
0.60 & 1.52413 & 1.52500 & 0.00087 & 0.00226 & 0.22636\\
0.65 & 1.54415 & 1.55192 & 0.00777 & 0.00328 & 0.32873\\
\end{tabular}
\end{table} 

\begin{table}[H]\caption{The calculated energies, $\left<E_L\right>$, for ten particle in three dimensions compared with the exact energy, $E_{ex}$. Both energies are of units $\hbar\omega_{oh}$. These calculations were performed with $2^{24}$ number of MC cycles. The normal standard deviation, $\sigma$, and the standard deviation from the blocking resampling method, $\sigma_B$ are also included.}\label{tab:brute_force_N_10_MC_boost}
\center
\begin{tabular}{cccccc}
$\alpha$ & $\left< E_L \right>$ & $E_{ex}$ & |$\left< E_L \right>-E_{ex}$|  & $\sigma_B$ & $\sigma$\\ \hline
0.35 & 15.85498 & 15.96429 & 0.10931 & 0.04753 & 1.35679\\
0.40 & 15.43695 & 15.37500 & 0.06195 & 0.03140 & 0.89352\\
0.45 & 15.10239 & 15.08333 & 0.01906 & 0.01406 & 0.41717\\
0.50 & 15.00000 & 15.00000 &                &                &                \\
0.55 & 15.05625 & 15.06818 & 0.01193 & 0.01224 & 0.37645\\
0.60 & 15.19959 & 15.25000 & 0.05041 & 0.02301 & 0.71856\\
0.65 & 15.58399 & 15.51923 & 0.06476 & 0.02798 & 1.02971\\
\end{tabular}
\end{table} 

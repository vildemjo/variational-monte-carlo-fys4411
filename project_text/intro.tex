For this project in FYS4411, Computational Physics II: Quantum Mechanical Systems, we are using Variational Monte Carlo (VMC) methods to study the ground state of a Bose-Einstein gas. This report includes a short explanation of how we have modelled the system with different approximations and potentials. Furthermore a short description of the different computational methods are included. Here the methods are explained using this specific problem as an example, to grasp the different methods. Finally, in the theoretical part of the report, the statistical aspects of the VMC method are explored to be able to analyse the results and get a good error estimation.

The results are shown with an escalating amount of complexity. At first, the simplest case is presented where we assume no interaction between the particles, we scan blindly over a range of parameters and we have no smart way of sampling the local energies. Gradually we include importance sampling, optimization and interaction while we compare the more complex situations with the starting point. Finally, we calculate the one-body density with and without interaction.